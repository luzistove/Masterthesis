\section {Basics}
\label{basics}
In this section, fundamentals of this thesis are introduced. First of all, the simulation environment, a multibody system which is used along this thesis is described, including the construction and the modelling foot-to-ground forces. This multibody system, representing the NAO robot, is built up in MATLAB Simscape. In the second main part, two kinds of inverse kinematics are presented, which is an important technique for finding joint angles in the field of robotics.

\subsection{The NAO Humanoid Robot}
The NAO humanoid robot, shown in Figure {\ref{NAO}}, is developed by Aldebaran Robotics and has been adopted as the football player in the RoboCup \ac{SPL} since 2008. The newest version of NAO so far is NAO V6 published in 2018. Possessing 25 \ac{DoFs} and being equipped with different kinds of sensors \cite{naosensorlist}, the NAO is able to perform human-like movements and interact with its surroundings.

The NAO robot is autonomous and programmable with various programming languages such as C++, Java and Python. As for simulation, there are also a wide range of software tools available for selection \cite{ivaldi2014tools}.
\begin{figure}[H]
	\centering
	\includegraphics[width=6cm]{fig/Nao.png}
	\caption[The NAO humanoid]{The NAO humanoid robot \cite{Naopicture}}
	\label{NAO}
\end{figure} 

\subsection{The Multibody System}
As shown in Figure {\ref{multibody}}, in this work the NAO robot is replaced by a multibody system in Simscape. Only the torso and both legs are constructed, while the arms and head are neglected. In a multibody system, solids in different geometrical shapes are connected by a variety of joints to build a mechanical system. Although being simple and with poor appearance, each part of the multibody system in Figure {\ref{multibody}} possesses the same mass, length and moment of inertia as that in the real NAO system. Such technical data can be found in the data sheet of Aldebaran Robotics \cite{Naoparameter} and are also listed in Appendix {\ref{appendixA}}. The masses of both arms and the head are merged into that of the upper torso.
\begin{figure}[H]
	\centering
	\includegraphics[width=8cm]{fig/multibody.jpg}
	\caption{The multibody system for simulation}
	\label{multibody}
\end{figure}

\subsubsection{Frames}

Each frame consists of three perpendicular axes that intersect at an origin. A frame is a significant characteristic affiliated to a solid, which directly determines its position and orientation in space by transition and rotation of certain values. Generally, it allows users to specify spatial relationships between different components \cite{frames}. When components are connected, actually their frames are connected; when measuring distances between objects, the distances of their frames are measured. Figure {\ref{robotframe}} shows the robot in a standing stance together with the frames of each component. 
\begin{figure}[H]
	\centering
	\includegraphics[width=8cm]{fig/robotframe.jpg}
	\caption{The multibody system and frames}
	\label{robotframe}
\end{figure}

\subsubsection{Joints}
To build a movable multibody system, solid are connected by proper joints to possess DoFs in space. For example, to model a simple cart-pendulum system, a revolution joint is attached between the cart and the rod. In the multibody system of a biped robot, there are joints at its hip, knee and ankle. The following types of joints are used according to the kinematic chain of the NAO robot shown in Figure {\ref{kinematicchain}}.
\begin{figure}[H]
	\centering
	\includegraphics[height=7cm]{fig/kinematicchain.png}
	\caption[The kinematic chain of NAO humanoid]{The kinematic chain of NAO humanoid \cite{gouaillier2009mechatronic}}
	\label{kinematicchain}
\end{figure}

\textbf{Universal joint}.
A universal joint provide two rotational DoFs. Bodies connected at both ends of a universal joint have free revolutional primitives about the $ x $ and $ y $ axis. Therefore, a leg is able to pitch and roll with a single universal joint at the hip. Similarly, there is also a universal joint at the ankle in order to achieve ankle pitch and roll motion.  

\textbf{Revolutional joint}.
A revolutional joint provides only one DoFs around its $ z $ axis. Thus it is used to realize the knee bending motion. As shown in Figure {\ref{kinematicchain}} (the blue cylinders at the hip), the hip yaw joint is rotated by \SI{45}{\degree} \cite{graf2009robust}, it is also modelled by a revolutional joint inclined by \SI{45}{\degree}.  

\textbf{Weld joint}.
There is a weld joint between the upper and lower torso, since they are together as an entirety on the real NAO robot.



\subsubsection{Foot-to-Ground Forces}
The foot-to-ground forces are playing an important role in the multibody system mentioned above. They are not only affecting its stability, but also the walking speed. With inappropriate setting of foot force, the multibody system cannot even stand on the ground. In this work, two foot-to-ground forces, the pressure $ \mathit{P} $ and friction $ \mathit{F_f} $, are modelled as contact forces of rigid bodies between tiny spheres on the four corners of the foot sole and a large plane, as shown in Figure {\ref{footforces}}. In simulation, the contact forces between two rigid bodies is realized with Simscape Contact Force Library from \cite{miller2018}.

\textbf{Pressure}. The pressure forces guarantee that the robot can stand and walk on ground without falling over in disturbance-free circumstances.
It is modelled as contact forces between eight tiny elastic spheres at the corner of both feet and a rigid plane \cite{guler1998viscoelastic}. As shown in Figure {\ref{footspringdamper}}, a elastic sphere can be viewed as a normal spring system enclosed by a spherical shell with radius $ r $. 
\begin{figure}[H]
	\centering
	\includegraphics[width=0.7\linewidth]{fig/footforce.pdf}
	\caption{Foot-to-ground forces}
	\label{footforces}
\end{figure}

The spring system is connected between the centre of the sphere and the ground, where the spring coefficient $ k $ is a parameter to be determined. When the foot is not treading on the ground, the spring system is not compressed.

During walking, it is possible that the robot is supported by only one of the eight elastic spheres. In this case, the total gravity is not distributed to the eight spheres but only one of them. Denoting the robot's total mass as $ m $ and the full compression of the spring system as $ r $, the spring coefficient of a single tiny sphere is calculated using Hooke's Law:
\begin{equation}
k_s = \frac{m_r\cdot g}{r},
\end{equation}
where $ g \approx$ \SI{9.81}{m/s^2} is the gravity acceleration. The radius of the tiny spheres is chosen as $ r = 10^{-5} $.


\begin{figure}[H]
	\centering
	\includegraphics[width=0.7\linewidth]{fig/footspring.pdf}
	\caption{Foot pressure model: a spring system}
	\label{footspringdamper}
\end{figure}



\textbf{Friction}. With proper friction the robot is able to walk on the  ground without sliding, which is of great significant for the gait simulation in the rest part of this thesis. Both static and kinetic friction coefficient have to be determined to model the friction forces between both feet and ground. However, since the materials of the foot as well as the ground are unknown in the multibody system, these two coefficient $ \mu_s $ and $ \mu_k $ are tuned roughly to find out the most appropriate combination. A simple gait is implemented to test how the walking speed is influenced by these two friction coefficients.

This simple gait is realized by periodically executing the predesigned joint angle sequences on hip, knee and ankle. Figure {\ref{anlgefrictiontest}} shows the joint angle motions of the left leg within \SI{1}{\second}, and those of the right leg are executed with a time delay of \SI{0.5}{\second}. Yaw and roll motion at the hips, as well as the roll motion at the ankles are deactivated so that the robot performs only straight walking.

\begin{figure}[H]
	\centering
	\includegraphics[width=\linewidth]{fig/anglefrictiontest.eps}
	\caption{Joint angle motion of the left leg for friction coefficients determination}
	\label{anlgefrictiontest}
\end{figure}

Average speed of walking is calculated by evaluating how far the robot can run within \SI{10}{\second}. Friction coefficients are tuned roughly. It can be observed from Table {\ref{simplegaitspeedtable}} that, when $ \mu_k $ and $ \mu_s $ is not less than 1000, the average speed reaches its maximum, approximately \SI{13.3}{cm/s} and this value maintains even though the coefficients keep on increasing. The combination of friction coefficients is determined within those that can result in the fastest average speed, which are marked with green. Table {\ref{simplegaitsimtimetable}} shows the measured elapse time of simulating the simple gait with different combinations of friction coefficients. 
\begin{table}[H]
	\centering
	\caption{Walking distance in \SI{10}{\second} of a simple gait with different friction coefficients}
	\label{simplegaitspeedtable}
	\begin{tabular}{lccccc}
		\hline
		&$\mu_k = 1\textsuperscript{*}$&$\mu_k = 10$&$\mu_k = 100$&$\mu_k = 1000$&$ \mu_k = 10000 $\\
		\hline
		$ \mu_s = 1\textsuperscript{**} $&\cellcolor{myred}0.250\textsuperscript{***}&\cellcolor{myred}0.465&\cellcolor{myred}0.522&\cellcolor{myred}0.530&\cellcolor{myred}0.530\\
		\hline
		$ \mu_s = 10 $&\cellcolor{myred}0.448&\cellcolor{myred}0.975&\cellcolor{myyellow}1.088&\cellcolor{myyellow}1.090&\cellcolor{myyellow}1.090\\
		\hline
		$ \mu_s = 100 $&\cellcolor{myred}0.468&\cellcolor{myyellow}1.297&\cellcolor{myyellow}1.299&\cellcolor{myyellow}1.299&\cellcolor{myyellow}1.299\\
		\hline
		$ \mu_s = 1000 $&\cellcolor{myred}0.473&\cellcolor{myyellow}1.326&\cellcolor{myyellow}1.330&\cellcolor{mygreen}1.331 &\cellcolor{mygreen}1.331\\
		\hline
		$ \mu_s = 10000 $&\cellcolor{myred}0.394&\cellcolor{myyellow}1.330&\cellcolor{myyellow}1.330&\cellcolor{mygreen}1.331&\cellcolor{mygreen}1.331\\
		\hline
		\multicolumn{6}{l}{\textsuperscript{*}\footnotesize{The kinetic friction coefficient $ \mu_k $ is dimensionless.}}\\
		\multicolumn{6}{l}{\textsuperscript{**}\footnotesize{The static friction coefficient $ \mu_s $ is dimensionless.}}\\
		\multicolumn{6}{l}{\textsuperscript{***}\footnotesize{The measured data in this table are length values with unit [\si{\meter}].}}
	\end{tabular}
\end{table}

\begin{table}[H]
	\centering
	\caption{Elapse time of a simple gait with different friction coefficients}
	\label{simplegaitsimtimetable}
	\begin{tabular}{lccccc}
		\hline
		&$\mu_k = 1\textsuperscript{*}$&$\mu_k = 10$&$\mu_k = 100$&$\mu_k = 1000$&$ \mu_k = 10000 $\\
		\hline
		$ \mu_s = 1\textsuperscript{**} $&23.39\textsuperscript{***}&24.80&26.59&35.36&51.10\\
		\hline
		$ \mu_s = 10 $&24.20&23.31&26.19&27.73&30.67\\
		\hline
		$ \mu_s = 100 $&24.76&25.59&25.03&26.14&27.55\\
		\hline
		$ \mu_s = 1000 $&26.60&27.01&26.28&\cellcolor{mygreen}26.98&28.15\\
		\hline
		$ \mu_s = 10000 $&29.93&29.09&27.77&28.94&29.27\\
		\hline
		\multicolumn{6}{l}{\textsuperscript{*}\footnotesize{The kinetic friction coefficient $ \mu_k $ is dimensionless.}}\\
		\multicolumn{6}{l}{\textsuperscript{**}\footnotesize{The static friction coefficient $ \mu_s $ is dimensionless.}}\\
		\multicolumn{6}{l}{\textsuperscript{***}\footnotesize{The measured data in this table are time values with unit [\si{\second}].}}
	\end{tabular}
\end{table}

Both kinetic and static friction coefficients are chosen as 1000 for the rest parts in this thesis.


\subsubsection{Finite State Machine}
A finite state machine is used for demonstrating the conditional, event-based and time-based logic and for modelling the interrupt mechanism provided by the CPU in the NAO. In this thesis, two states are responsible for the motion of the left and right leg respectively and they switch between each other when the state transition conditions are satisfied. As a central control unit, the finite state machine is in charge of planning and generating motion commands. It is realized by using Stateflow in Simulink.



\subsection{Inverse Kinematics}
\label{ik}
In robotics, an end effector is the device at the end of limbs, which is designed to interact with the surroundings. Inverse kinematics is a common technique in robot motion planning, aiming at finding the joint angles on a kinematic chain when the position and orientation of the end effectors are given\cite{spong2008robot}. When it comes to a humanoid robot, the end effectors are the hands and feet. In this thesis, the inverse kinematics technique is applied to determine joint angles on the lower body kinematic chain required for walking, which are listed in Table {\ref{Jointanglesabb}}.
\begin{table}[H]
	\centering
	\caption{Joint angles and abbreviations}
	\label{Jointanglesabb}
	\begin{tabular}{cc}
		\hline
		Joint angle & Abbreviation \\
		\hline
		hip pitch & $ h_P $ \\
		\hline
		hip roll & $ h_R $ \\
		\hline
		hip yaw & $ h_Y $\\
		\hline
		knee bend & $ k_B $ \\
		\hline 
		ankle pitch & $ a_P $ \\
		\hline
		ankle roll & $ a_R $ \\
		\hline
	\end{tabular}
\end{table}

\subsubsection{Closed-Form Inverse Kinematics}
In \cite{hengst2014runswift}, a closed-form inverse kinematic method is proposed by calculating the joint angles geometrically. First of all, some parameters of walking are defined in Table {\ref{motionparameter}}, and the length of the thigh and tibia are denoted as $ l_{th} $ and $ l_{ti} $ respectively.

\begin{table}[H]
	\centering
	\caption{Walking parameters and abbreviations}
	\label{motionparameter}
	\begin{tabular}{cc}
		\hline
		Walking parameters& Abbreviation \\
		\hline
		CoM offset in $ y $ direction & $ \mathit{CoM_{y}} $ \\
		\hline
		Foot lifting height& $ h_L $ \\
		\hline
		Forward step size & $ s_F $\\
		\hline
	\end{tabular}
\end{table}

Figure {\ref{jointangles}} shows the joint angles to be determined with closed-form inverse kinematics. The hip roll angle, hip pitch angle and knee bend angle can be achieved geometrically with the parameters listed in Table {\ref{motionparameter}}:
\begin{figure}[H]
	\centering
	\includegraphics[width=0.45\linewidth]{fig/jointangle.pdf}
	\caption{Joint angles to be determined with closed-form inverse kinematics}
	\label{jointangles}
\end{figure}

\begin{align}
	\begin{split}
		h_R &= \arcsin \frac{\mathit{CoM_y}}{l_{th}+l_{ti}}\\
		h_P &= \arctan\frac{s_F}{\left(l_{th}+l_{ti}-\frac{h_L}{\cos h_R}\right)}+\arccos \frac{l_{th}^2+s_F^2+\left(l_{th}+l_{ti}-\frac{h_L}{\cos h_R}\right)^2-l_{ti}^2}{2\cdot l_{th}\cdot \sqrt{s_F^2+\left(l_{th}+l_{ti}-\frac{h_L}{\cos h_R}\right)^2}}\\
		k_B &= \pi - \arccos \frac{l_{th}^2+l_{ti}^2-\left(s_F^2+\left(l_{th}+l_{ti}-\frac{h_L}{\cos h_R}\right)^2\right)}{2\cdot l_{th}\cdot l_{ti}}\\
	\end{split}
\end{align}

Both feet are kept parallel to the ground in the whole process of walking and therefore, ankle pitch and roll angles can be derived from calculated hip and knee angles.
\begin{align}
\label{anklejoint}
	\begin{split}
		a_P &= h_P+k_B\\
		a_R &= -h_R\\
	\end{split}
\end{align}


\subsubsection{Iterative Inverse Kinematics}
Another inverse kinematics method is to search the joint angles iteratively. 
The kinematic chain is analyzed by take the left leg as an example while it is the same on the right leg except for the hip offset in an opposite direction. 	
As the hip, knee and ankle joint possess respectively 3, 1 and 2 degrees of freedom, first define the joint angles as following: 


Given the coordinate of the CoM in space, the kinematic chain generates that of the midpoint of the foot sole. The task of inverse kinematics is to calculate the joint angles in table {\ref{Joint angles abb}} at each time step to place the foot at the desired location, which is predefined.


The kinematic chain of the lower body on both planes is shown in Figure {\ref{kinchainlow}}, which is depicting a single support gesture with the left leg lifting. The crucial joints and points are emphasized by several black dots. The frames noted with $ x $, $ y $ and $ z $ are the global frames.

Transformation between frames in space, including rotation and translation can be treated as a series of matrix multiplication.
Basic homogeneous transformation matrices representing translations of distance $ d $ along $ x $, $ y $ and $ z $ axis and rotations of degree $ \theta $ about $ x $, $ y $ and $ z $ axis respectively are given as \cite{spong2008robot}:

\begin{equation}
T_{x}(d)=
\begin{bmatrix}
1&0&0&d\\0&1&0&0\\0&0&1&0\\0&0&0&1
\end{bmatrix},
\quad
R_{x}(\theta)=
\begin{bmatrix}
1&0&0&0\\0 &\cos\theta&-\sin\theta&0\\0 &\sin\theta&\cos\theta&0\\0&0&0&1
\end{bmatrix};
\end{equation}

\begin{equation}
T_{y}(d)=
\begin{bmatrix}
1&0&0&0\\0&1&0&d\\0&0&1&0\\0&0&0&1
\end{bmatrix},
\quad
R_{y}(\theta)=
\begin{bmatrix}
\cos\theta&0&\sin\theta&0\\0&1&0&0\\-\sin\theta&0&\cos\theta&0\\0&0&0&1
\end{bmatrix};
\end{equation}

\begin{equation}
T_{z}(d)=
\begin{bmatrix}
1&0&0&0\\0&1&0&0\\0&0&1&d\\0&0&0&1
\end{bmatrix},
\quad
R_{z}(\theta)=
\begin{bmatrix}
\cos\theta&-\sin\theta&0&0\\\sin\theta&\cos\theta&0&0\\0&0&1&0\\0&0&0&1
\end{bmatrix}.
\end{equation}
As the kinematic chain of the lower body starts from the CoM and ends at the midpoint of the foot sole, a homogeneous transformation matrix $ H $ that represents this kinematic chain is calculated with the help of Denavit-Hartenberg Convention. The frame of each joint are chosen as in Figure {\ref{kinchainlow}}, so that the $ z $ axes coincide with the rotation axes of the joints and the directions of $ x $ axes follow those of the links.

\begin{table}[H]
	\centering
	\label{DHlow}
	\caption{Denavit-Hartenberg parameters for the lower body kinematic chain}
	\begin{tabular}{ccccc}
		\hline
		Link&$ a_i $&$ \alpha_i $&$  d_i$&$ \theta_i $\\
		\hline
		1&$ \frac{1}{2}\mathit{hipOffset_z} $&0&0&0\\
		\hline
		2&$ \mathit{hipOffset_y} $&0&0&$ \frac{\pi}{2} $\\
		\hline
		3&0&$ -\frac{\pi}{2} $&0&$ -\frac{\pi}{2}+h_r $\\
		\hline
		4&0&0&0&$ h_p $\\
		\hline
		5&$ l_{thigh} $&0&0&$ k_b $\\
		\hline
		6&$ l_{tibia} $&0&0&$ a_p $\\
		\hline
		7&$ h_{foot} $&0&0&$ a_r $\\
		\hline
		8&$ \frac{1}{4}l_{foot} $&0&0&$ -\frac{\pi}{2} $\\
		\hline
	\end{tabular}
\end{table}

In Denavit-Hartenberg Convention, the homogeneous transformation matrix that transforming the coordinate from joint $ i-1 $ to joint $ i $ which are on the two ends of link $ i $, are defined as:
\begin{equation}
A_i=R_z(\theta_i)T_z(d_i)T_x(a_i)R_x(\alpha_i),\,i=1,\cdots, 8.
\end{equation}
And the homogeneous transformation matrix of the whole kinematic chain $ H $ is given by:
\begin{equation}
\label{H}
H=A_1A_2\cdots A_8.
\end{equation}
Once having the coordinate of the CoM, that of the foot sole midpoint can be simply calculated with:
\begin{equation}
\label{comsole}
C_{sole}= HC_{CoM},
\end{equation}
where $ C_{CoM} $ and $ C_{sole} $ are the augmented CoM and foot sole midpoint coordinates considering a scaling factor, which is chosen as 1 generally:
\begin{equation*}
C_{CoM}=
\begin{bmatrix}
x_{CoM}\\y_{CoM}\\z_{CoM}\\1
\end{bmatrix},
\quad
C_{sole}=
\begin{bmatrix}
x_{sole}\\y_{sole}\\z_{sole}\\1
\end{bmatrix}.
\end{equation*}
However, the coordinate of the foot sole midpoint in Equation {\ref{comsole}} is based on the frame which is initially selected, instead of the global one. In order to obtain a foot sole midpoint coordinate based on a global frame, another rotation of $ 90^{\circ} $ about the $ y $ axis has to be performed.

\subsubsection{Damped Least Squares Method}
The transformation matrix $ H $ in Equation {\ref{H}} is parametrized by unknown joint angles in Table {\ref{Joint angles abb}}, some entries will include complicated and nonlinear sine and cosine multiplications, which are difficult to solve for a solution of: 
\begin{equation*}
\{h_p,h_r,h_y,k_b,a_p,a_r\}
\end{equation*}
is not explicit and simple to find with the given $ C_{CoM} $ and $ C_{sole} $, even if there exist. An alternative is to search these joint angles iteratively by initially guessing some joint angles and then updating them until the end effectors reach the desired places. 

A damped least squares method is a common technique for solving inverse kinematics problems
\cite{wampler1986manipulator}
\cite{buss2004introduction} and was used in the walking engine 
\cite{hengst2014runswift}for the turning motion. Here in this work it is used for the whole phase of walking.

Supposed a multibody system with $ k $ end effectors and $ n $ joints are being controlled so that they are aimed to be place at target locations 
$ \bm{r}= \begin{bmatrix}
r_1,\cdots, r_k
\end{bmatrix}^\prime$. The current position of the end effectors 
$ \bm{p}=\begin{bmatrix}
p_1,\cdots, p_k
\end{bmatrix}^\prime $ are given by the joint angles 
$ \bm{\theta}=
\begin{bmatrix}
\theta_1, \cdots, \theta_n
\end{bmatrix}^\prime $. Therefore, the position of end effectors is a function of joint angles:
\begin{equation}
\bm{p}=f(\bm{\theta}).
\end{equation}
Let $ \bm{e} $ denote the error between the target and the current position 
\begin{equation}
\bm{e} = \bm{r} -\bm{p},
\end{equation}
the damped least squares method is implemented as in the following algorithm:

\begin{algorithm}[H]  
	\label{DLSM}
	\caption{Damped Least Squares Method $(\bm{r}, H, \lambda, \mathit{tol})$}
	\begin{algorithmic}[1]
		\STATE \text{Initial guess} $ \bm{\theta} $ 
		\STATE $ \bm{p}=f(\bm{\theta}) $
		\STATE $ \bm{e}=\bm{r}-\bm{p} $
		\REPEAT 
		\STATE	$ \Delta\bm{\theta}=J^T(JJ^T+\lambda^2I)^{-1}\bm{e} $
		\STATE $ \bm{\theta} \gets \bm{\theta}+\Delta\bm{\theta}$
		\STATE $ \bm{p}=f(\bm{\theta}) $
		\STATE $ \bm{e}=\bm{r}-\bm{p} $
		\UNTIL $ \left\Vert\bm{e}\right\Vert \leqslant tol$
	\end{algorithmic}  
\end{algorithm}
where $ \lambda $ is the damping coefficient and $ J $ is the Jacobian matrix of $ \bm{\theta} $ with
\begin{equation}
J(\bm{\theta})=(\frac{\partial p_i}{\partial \theta_j})_{i,j}, \quad i = 1, \cdots, k, \, j = 1, \cdots, n.
\end{equation}

